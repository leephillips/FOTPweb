\documentclass{article}
\usepackage[absolute]{textpos}
% \usepackage{multicol}
\usepackage[cm-default]{fontspec}
% \defaultfontfeatures{Ligatures=TeX}
\usepackage[margin=0.25in]{geometry}
\usepackage{graphicx}
 % \usepackage{floatflt}
\usepackage{wrapfig}
\setlength{\TPHorizModule}{1in}
\setlength{\TPVertModule}{\TPHorizModule}
\textblockorigin{0.25in}{0.25in} % start everything near the top-left corner
\setlength{\parindent}{0pt}
\setlength{\intextsep}{0pt}
\setlength{\columnsep}{3pt}
% \begin{minipage}[pos][height][contentpos]{width} text \end{minipage}
\begin{document}
\includegraphics[width=\textwidth]{../head1.png}
% \begin{textblock}{7.5}(0,2.5)
   {\fontsize{28}{8}\fontspec{Impact}
   \begin{centering}

   November 15–17 2003: The Comet is Here!

   \end{centering}
}
   

\begin{textblock}{3.9}(0,2.8)

   {\fontsize{19}{21}\fontspec{Liberation Sans Narrow}

   \begin{wrapfigure}{l}{0.9in}

          \includegraphics[width=0.8in]{Magic_treehouse_poster}
       
       \end{wrapfigure}      

       \textbf{Magic Tree House® Space Mission – Fri., Nov.~15, 6:30 pm}: Travel with the brother-sis\-ter duo, 
       Jack and
       Annie in their Magic Tree House as they pro\-ceed to an\-swer the questions left
       for them in a mysterious note signed —M. 
       
       }

\end{textblock}

\begin{textblock}{4.0}(0,5)
   \begin{wrapfigure}[12]{l}{2in}
      \includegraphics[width=1.9in]{Herschel}
   \end{wrapfigure}      

   {\fontsize{20}{32}\fontspec{Linux Biolinum O}
   \textbf{The “Ladie’s Com\-et” – Fri., Nov.~15, 7:30 pm}:
   An even\-ing with 18$^{\hbox{th}}$ century celestial siblings, Wil\-liam and Caroline
   Herschel. 
   % The first bro\-ther and sister as\-tro\-no\-mers, the Herschel’s
   % discoveries include binary stars, moons, nebulae, a new planet, and 8
   % comets. As a comet hunter, Caroline blazed a trail for future female
   % scientists. 

   }

\end{textblock}


 \begin{textblock}{3.2}(4.5,5.8)

   {\fontsize{16}{17}\fontspec{Sawasdee}
   % This program explores the relationship between the Earth, Moon and Sun with
   % the help of Coyote, an amusing character adapted from Native American oral
   % traditions who has many misconceptions about our home planet and its most
   % familiar neighbors.
   % \begin{wrapfigure}{l}{2.5in}

   \includegraphics[width=3.2in]{EarthMoonSun}

   \textbf{Sun., Nov.~17, 1:30 pm}
   % \end{wrapfigure}
   % His confusion about the universe makes viewers think
   % about how the Earth, Moon, and Sun work together as a system. Native
   % American stories are used throughout the show to help distinguish between
   % myths and science. Target audience ages 7 and up.

   }

\end{textblock}

\begin{textblock}{3.2}(4.5,2.8)

   \includegraphics[width=3.2in]{MST3k_from_Hulu}

   {\fontsize{16}{17}\fontspec{LMRoman10}
   \textbf{“Movie Night at the Planetarium”, Sat., Nov.~16, 6:30 pm}:
    A surprise Mystery Science Theater 3000 (MST3k) presentation! 
    \textbf{Rated TV14 for Language.}

    }

\end{textblock}

\begin{textblock}{3.5}(4.2,7.4)
   {\fontsize{16}{17}\fontspec{Trebuchet MS}

   \begin{centering}

   An Interview with a Mad Scientist 

   \end{centering}

   \includegraphics[width=3.5in]{SupermanLois}

   Reporter Lois Lane hosts a double-feature of classic Superman cartoons. 

   }
\end{textblock}

   {\fontsize{16}{17}\fontspec{Trebuchet MS}

\begin{textblock}{1.2}(6.0,7.7)
   \textbf{Sun., Nov. 17,}
\end{textblock}
\begin{textblock}{2}(6.9,7.9)
   3 pm
\end{textblock}

   }


\begin{textblock}{3.6}(0,8.2)
   {\fontsize{16}{17}\fontspec{LMRomanDunh10}

Admission for all programs is \$3 for Children (up to age 12), \$5 for Friends, Members, and Seniors (60+), and \$7 for teens and adults.

   }
\end{textblock}

\begin{textblock}{3}(0,9.7)

   {\fontsize{18}{19}\fontspec{LMRoman10}
\begin{wrapfigure}{l}{0.7in}
   \includegraphics[width=0.7in]{QRnov2013}
\end{wrapfigure}      

For tickets and more information, go to http://friendsoftheplanetarium.org.

}
\end{textblock}

\end{document}



