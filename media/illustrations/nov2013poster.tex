\documentclass{article}
% \usepackage[absolute]{textpos}
\usepackage{multicol}
\usepackage[cm-default]{fontspec}
% \defaultfontfeatures{Ligatures=TeX}
\usepackage[margin=0.25in]{geometry}
\usepackage{graphicx}
 \usepackage{floatflt}
% \usepackage{wrapfig}
% \setlength{\TPHorizModule}{1in}
% \setlength{\TPVertModule}{\TPHorizModule}
% \textblockorigin{10mm}{10mm} % start everything near the top-left corner
\setlength{\parindent}{0pt}
% \begin{minipage}[pos][height][contentpos]{width} text \end{minipage}
\begin{document}
% \begin{textblock}{7.5}(0,0)
\includegraphics[width=\textwidth]{../head1.png}
% \end{textblock}
% \begin{textblock}{7.5}(0,2.5)
   {\fontsize{30}{40}\fontspec{DejaVuSans}
   \begin{centering}
   NOVEMBER 2013 PROGRAMMING

   \end{centering}
   
   }

\begin{multicols}{2}

   {\fontsize{18}{20}\fontspec{Gentium}

   The theme for the November 15 — 17 weekend is \textbf{The Comet is Here!}.

}

   \vskip 0.4cm

   \begin{floatingfigure}{2in}
      \includegraphics[width=1.9in]{Herschel}
   \end{floatingfigure}      

   \textbf{The “Ladie’s Comet” — Fri., Nov.~15, 7:30 pm}:
   An evening with eighteenth century celestial siblings, William and Caroline Herschel. The first brother and sister astronomers, the Herschel’s discoveries include binary stars, moons, nebulae, a new planet, and 8 comets. As a comet hunter, Caroline blazed a trail for future female scientists. Hear of their remarkable journey from accomplished musicians to famous astronomers. Target audience ages 8 and up.

   \vskip 0.4cm

   \includegraphics[width=3.8in]{EarthMoonSun}

   \textbf{Earth, Moon \& Sun: Sun., Nov.~17, 1:30 pm}:
   This program explores the relationship between the Earth, Moon and Sun with the help of Coyote, an amusing character adapted from Native American oral traditions who has many misconceptions about our home planet and its most familiar neighbors. His confusion about the universe makes viewers think about how the Earth, Moon, and Sun work together as a system. Native American stories are used throughout the show to help distinguish between myths and science. Target audience ages 7 and up.

\columnbreak



   {\fontsize{18}{20}\fontspec{Liberation Sans Narrow}

       Magic Tree House® Space Mission — Travel with the brother-sis\-ter duo, 
       \begin{floatingfigure}{0.9in}

          \includegraphics[width=0.9in]{Magic_treehouse_poster}
       
       \end{floatingfigure}      
       Jack and
       Annie in their Magic Tree House as they pro\-ceed to an\-swer the questions left
       for them in a mysterious note signed —M. 
       
       }

   \vskip 0.4cm

   \textbf{“Movie Night at the Planetarium”, Sat., Nov.~16, 6:30pm}:
   \includegraphics[width=3.5in]{MST3k_from_Hulu}
    A surprise Mystery Science Theater 3000 (MST3k) presentation! 
    Rated TV14 for Language.

   \vskip 0.4cm

   \includegraphics[width=3.5in]{SupermanLois}
   \textbf{An Interview with a Mad Scientist: Sun., Nov.~17, 3pm}:
   Reporter Lois Lane hosts a double-feature of classic Superman cartoons featuring thrilling clashes with evil scientists bent on destruction. Lois will interview one fiendish scientist who thinks that he and his comrades are victims of bad publicity! Suitable for all ages.

\end{multicols}

Admission for all programs is \$3 for Children (up to age 12), \$5 for Friends, Members, and Seniors (60+), and \$7 for teens and adults.

\begin{floatingfigure}{1.3in}
   \includegraphics[width=1.1in]{QRnov2013}
\end{floatingfigure}      

For tickets and more information, go to http://friendsoftheplanetarium.org

\end{document}



