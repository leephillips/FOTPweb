\documentclass{article}
% \usepackage[absolute]{textpos}
\usepackage{multicol}
\usepackage[cm-default]{fontspec}
% \defaultfontfeatures{Ligatures=TeX}
\usepackage[margin=0.25in]{geometry}
\usepackage{graphicx}
 % \usepackage{floatflt}
\usepackage{wrapfig}
% \setlength{\TPHorizModule}{1in}
% \setlength{\TPVertModule}{\TPHorizModule}
% \textblockorigin{0.125in}{0.125in} % start everything near the top-left corner
\setlength{\parindent}{0pt}
\setlength{\intextsep}{0pt}
\setlength{\columnsep}{0.24in}
% \setlength{\linewidth}{3in}
% \begin{minipage}[pos][height][contentpos]{width} text \end{minipage}

% 1/8 in margins with 2 col and 1/4 in space between col = colwidth of 4 in,
% 2nd col begins at at 4.25 with origin at 1/8.
% Reduce col width to 3.8 in to handle overfulls.

\begin{document}
\includegraphics[width=\textwidth]{../media/NewDMBHeader}
% \begin{textblock}{7.5}(0,2.5)
   \begin{centering}
     {\fontsize{28}{8}\fontspec{STIXGeneral}
  
        June 17 – 19, 2016: PLANETS!
     }

     {\fontsize{19}{19}\fontspec{Liberation Sans Narrow}
      
        A weekend at the Arlington Planetarium\break devoted to 
        planets in the solar system and beyond.

     }
  
   \end{centering}

% \begin{multicols}{2}

     % {\fontsize{20}{22}\fontspec{STIXGeneral}
   {\fontsize{19}{19}\fontspec{Liberation Sans Narrow}

     \begin{wrapfigure}[5]{l}{0pt}
       \includegraphics[width=3.8in]{dynamicEarth2}
     \end{wrapfigure}      
     
      \textbf{Fri  6:30 pm}: Ride along on swirling ocean and wind currents, dive into the heart of a monster hurricane, come face-to-face with sharks and gigantic whales, and fly into roiling volcanoes in this full-dome movie.

   % {\fontsize{20}{24}\fontspec{STIXGeneral}

   \begin{wrapfigure}[6]{l}{0pt}
      \includegraphics[width=1.0in]{../media/illustrations/michaelNeufeld}
   \end{wrapfigure}      
   
   \textbf{Michael Neufeld: The New Horizons Mission to Pluto: Fri, 7:30 pm}: 
      The story of one of the most ambitious planetary flyby missions ever attempted: the New Horizons mission to Pluto,
      told by a Senior Curator at the Smithsonian Air \& Space Museum.
      


 % \begin{wrapfigure}[6]{r}{0pt}
 %   \includegraphics[width=1.0in]{../media/illustrations/Exploding_Universe.jpeg}
 % \end{wrapfigure}      


     \textbf{Exploding Universe: Sat, 6:30}: From supernovas to volcanoes, this film looks at the beautiful and creative power of explosive phenomena to shape planets, stars, and the entire universe.



 \begin{wrapfigure}[6]{l}{0pt}
   \includegraphics[width=1.0in]{HHammel561}
 \end{wrapfigure}      

   \textbf{Heidi Hammel Lecture: “Exploring the Solar System with the Hubble Space Telescope”: Sat. 7:30 pm}. 
Astronomer Heidi Hammel has been profiled by the New York Times and Newsweek Magazine, and was identified as one of the 50 most important women in science by Discover Magazine in 2002. She’ll show us colliding comets, active aurorae, Martian storms, and much more.


\textbf{Sunday: Two popular movies for young people:} “Perfect Little Planet” at 1:30 and “Magic Tree House: 
Space Mission” at 3:00. Plus a free hands-on educational activity between the shows!

}

   {\fontsize{16}{18}\fontspec{Ubuntu Mono}

Admission for all programs is \$3 for children ($\le$12) and seniors ($\ge$60) and \$5 for all others.

   }
% \end{textblock}

% \begin{textblock}{7}(0.65,10)


   {\fontsize{18}{24}\fontspec{Ubuntu Mono}

\begin{wrapfigure}{r}{0.8in}
   \includegraphics[width=0.8in]{QRplanets2016.jpg}
\end{wrapfigure}      

\begin{centering}
  
For tickets and more information, go to\break
http://friendsoftheplanetarium.org.

\end{centering}

}



% \end{textblock}
\end{document}



